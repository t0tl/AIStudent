\documentclass{beamer}
\usepackage{amsmath}
\usepackage{graphicx}
\usepackage{hyperref}

\title{Lecture: Data Analysis and Machine Learning Theory}
\author{KTH AI Student}
\date{\today}

\begin{document}

\frame{\titlepage}

\section{Introduction to \texttt{uv}}
\begin{frame}{Installing Required Packages with \texttt{uv}}
\begin{itemize}
    \item \texttt{uv}: Modern tool for managing virtual environments.
    \item Features:
    \begin{itemize}
        \item Inline dependency management: Specify dependencies directly in your code for better reproducibility.
        \item Faster installations: Uses efficient caching to minimize installation time.
        \item Lockfiles: Ensures consistent environments across systems by locking dependency versions.
    \end{itemize}
    \item Installation: \texttt{pip install uv}
    \item Usage: \texttt{uv run <name\_of\_file.py>}, automatically handles dependencies.
\end{itemize}
\end{frame}

\section{Descriptive Statistics and Data Visualization}

\begin{frame}{Example: Student Test Scores}
    \begin{itemize}
        \item \textbf{Dataset:} Contains scores of students.
        \item \textbf{Goals:}
        \begin{itemize}
            \item Compute key descriptive statistics to summarize performance.
            \item Visualize score distributions to identify trends or outliers.
            \item Provide actionable insights to improve teaching methods.
        \end{itemize}
    \end{itemize}
    \end{frame}

\begin{frame}{Key Concepts}
\textbf{Descriptive Statistics:} Summarize and describe the main features of a dataset.
\begin{itemize}
    \item \textbf{Mean}: The average value of a dataset.
    $$ \bar{x} = \frac{1}{n} \sum_{i=1}^{n} x_i $$
    \item \textbf{Median}: The middle value when data is sorted.
    $$ x_{\text{median}} = \begin{cases} x_{(n+1)/2} & \text{if } n \text{ is odd} \\ \frac{1}{2} (x_{n/2} + x_{n/2+1}) & \text{if } n \text{ is even} \end{cases} $$
    \item \textbf{Mode}: The most frequently occurring value.
    $$ x_{\text{mode}} = \text{value with highest frequency} $$
\end{itemize}
\end{frame}

\begin{frame}{Key Concepts}
\begin{itemize}
    \item \textbf{Variance}: Measures the spread of data points from the mean.
    $$ \sigma^2 = \frac{1}{n} \sum_{i=1}^{n} (x_i - \bar{x})^2 $$
    \item \textbf{Standard Deviation}: Square root of variance, represents data dispersion.
    $$ \text{SD} = \sqrt{\sigma^2} $$
    \item \textbf{Range}: Difference between the maximum and minimum values.
    $$ \text{Range} = \max(x) - \min(x) $$
\end{itemize}
\end{frame}

\begin{frame}{Key Concepts}
\begin{itemize}
    \item \textbf{Skewness}: Measures asymmetry of data distribution.
    $$ \text{Skewness} = \frac{\frac{1}{n} \sum_{i=1}^{n} (x_i - \bar{x})^3}{\left(\frac{1}{n} \sum_{i=1}^{n} (x_i - \bar{x})^2\right)^{3/2}} $$
    \item \textbf{Kurtosis}: Measures the \textit{tailedness} of the data distribution.
    $$ \text{Kurtosis} = \frac{\frac{1}{n} \sum_{i=1}^{n} (x_i - \bar{x})^4}{\left(\frac{1}{n} \sum_{i=1}^{n} (x_i - \bar{x})^2\right)^2} $$
\end{itemize}
\end{frame}

\begin{frame}{Key Concepts}
\textbf{Data Visualization:} Graphical representation of data.
\begin{itemize}
    \item \textbf{Histograms}: Show frequency distribution of data.
        \begin{figure}
            \centering
            \includegraphics[width=0.5\textwidth]{histogram.png}
        \end{figure}
    \item \textbf{Box Plots}: Visualize data spread and identify outliers.
        \begin{figure}
            \centering
            \includegraphics[width=0.5\textwidth]{box-plot.png}
        \end{figure}
    \end{itemize}
\end{frame}

\begin{frame}{Key Concepts}
\begin{itemize}
    \item \textbf{Scatter Plots}: Display relationships between two variables.
    \begin{figure}
        \centering
        \includegraphics[width=0.5\textwidth]{scatter-plot.png}
    \end{figure}
    \item \textbf{Bar Charts}: Compare categorical data.
    \begin{figure}
        \centering
        \includegraphics[width=0.3\textwidth]{bar-chart.png}
    \end{figure}
\end{itemize}
\end{frame}


\section{Probability Theory and Simulation}


\begin{frame}{Example: Simulation Tasks}
    \begin{itemize}
        \item Simulate 1000 coin tosses to calculate the probability of heads and compare with theoretical value.
        \item Simulate 1000 dice rolls to calculate:
        \begin{itemize}
            \item Probability of rolling a prime number.
            \item Conditional probability of a prime given the number is odd.
        \end{itemize}
        \item Use Monte Carlo simulation to estimate $\pi$.
    \end{itemize}
    \end{frame}


\begin{frame}{Key Concepts: Probability}
\begin{itemize}
    \item \textbf{Probability:} Study of the likelihood of events.
    \begin{itemize}
        \item \textbf{Theoretical Probability:} Based on known outcomes (e.g., coin toss).
        $$ P(A) = \frac{\text{Number of favorable outcomes}}{\text{Total number of outcomes}} $$
        \item \textbf{Simulated Probability:} Estimated by running experiments or simulations.
        \item \textbf{Bayes' Theorem:} Describes conditional probability, updates beliefs based on evidence.
        $$ P(A|B) = \frac{P(B|A) \cdot P(A)}{P(B)} $$
    \end{itemize}
\end{itemize}
\end{frame}

\begin{frame}{Key Concepts: Probability Distributions}
\begin{itemize}
    \item \textbf{Probability Distributions:} Represent how probabilities are distributed over values.
    \begin{itemize}
        \item \textbf{Uniform Distribution:} All outcomes are equally likely.
        $$ P(x) = \frac{1}{n} \quad \text{for } x \in \{1, 2, \ldots, n\} $$
        \item \textbf{Binomial Distribution:} Number of successes in fixed trials.
        $$ P(X = k) = \binom{n}{k} p^k (1-p)^{n-k} $$
        \item \textbf{Normal Distribution:} Bell-shaped curve, common in natural data.
        $$ f(x) = \frac{1}{\sqrt{2\pi\sigma^2}} e^{-\frac{(x-\mu)^2}{2\sigma^2}} $$
    \end{itemize}
\end{itemize}
\end{frame}

\begin{frame}{Key Concepts: Monte Carlo Simulation}
\begin{itemize}
    \item \textbf{Monte Carlo Simulation:} Uses random sampling to estimate mathematical results.
    \begin{itemize}
        \item Example: Estimate $\pi$ by generating random points in a square and calculating the ratio inside a quarter circle.
        $$ \pi \approx 4 \times \frac{\text{Number of points inside circle}}{\text{Total number of points}} $$
    \end{itemize}
\end{itemize}
\end{frame}

\section{Correlation and Regression Analysis}

\begin{frame}{Key Concepts: Correlation}
\begin{itemize}
    \item \textbf{Correlation:} Measures linear relationship strength and direction.
    \begin{itemize}
        \item Coefficient values range from -1 (perfect negative) to 1 (perfect positive).
        \item Value near 0 implies weak or no linear relationship.
    \end{itemize}
\end{itemize}
\end{frame}

\begin{frame}{Key Concepts: Regression Analysis}
\begin{itemize}
    \item \textbf{Regression Analysis:} Models the relationship between variables.
    \begin{itemize}
        \item \textbf{Simple Linear Regression:} $y = \beta_0 + \beta_1 x + \epsilon$
        \item Coefficients ($\beta_0$, $\beta_1$) are estimated to minimize error ($\epsilon$).
        \item \textbf{Evaluation Metrics:} Mean Squared Error (MSE), measures average squared differences between actual and predicted values.
    \end{itemize}
\end{itemize}
\end{frame}

\begin{frame}{Example: Car Prices and Mileage}
\begin{itemize}
    \item \textbf{Dataset:} Car prices and mileage.
    \item \textbf{Goals:}
    \begin{itemize}
        \item Calculate correlation to determine strength of relationship.
        \item Build a regression model to predict car prices based on mileage.
        \item Visualize data points and regression line to assess model fit.
    \end{itemize}
\end{itemize}
\end{frame}

\section{A/B Testing and Hypothesis Testing}
\begin{frame}{Key Concepts}
\begin{itemize}
    \item \textbf{Hypothesis Testing:} Framework to evaluate evidence against a null hypothesis ($H_0$).
    \begin{itemize}
        \item \textbf{Null Hypothesis ($H_0$):} Assumes no effect or difference.
        \item \textbf{Alternative Hypothesis ($H_a$):} Suggests a significant effect or difference.
    \end{itemize}
    \item \textbf{t-Test:} Compares means of two groups to assess differences.
    \begin{itemize}
        \item \textbf{t-statistic:} Measures difference relative to data variability.
        \item \textbf{p-value:} Probability of observing results as extreme as current data under $H_0$.
    \end{itemize}
\end{itemize}
\end{frame}

\begin{frame}{Example: Website Redesign}
\begin{itemize}
    \item \textbf{Dataset:} Engagement metrics for old and new website designs.
    \item \textbf{Goals:}
    \begin{itemize}
        \item Perform t-test to compare user engagement.
        \item Interpret p-value to determine statistical significance.
        \item Provide actionable recommendations based on findings.
    \end{itemize}
\end{itemize}
\end{frame}

\section{Linear Regression and Gauss-Markov Assumptions}
\begin{frame}{Key Concepts}
\begin{itemize}
    \item \textbf{Linear Regression:} Predicts a dependent variable from independent variable(s).
    \item \textbf{Gauss-Markov Assumptions:}
    \begin{itemize}
        \item \textbf{Linearity:} Relationship between predictors and outcome is linear.
        \item \textbf{Independence:} Residuals (errors) are independent of each other.
        \item \textbf{Homoscedasticity:} Residuals have constant variance.
        \item \textbf{No Multicollinearity:} Independent variables are not highly correlated.
        \item \textbf{Normality of Errors:} Errors are normally distributed (optional).
    \end{itemize}
    \item \textbf{Implications of Violations:} Leads to biased estimates, inefficient predictions, or unreliable inference.
\end{itemize}
\end{frame}

\begin{frame}{Example: Predicting Housing Prices}
\begin{itemize}
    \item \textbf{Dataset:} House prices based on square footage.
    \item \textbf{Goals:}
    \begin{itemize}
        \item Build regression model to predict house prices.
        \item Evaluate Gauss-Markov assumptions (e.g., residual analysis).
        \item Interpret coefficients and assess model validity.
    \end{itemize}
\end{itemize}
\end{frame}

\section{Conclusion}
\begin{frame}{Summary}
\begin{itemize}
    \item Reviewed key concepts in data analysis and machine learning:
    \begin{itemize}
        \item Descriptive statistics: Summarized data distributions.
        \item Probability: Explored theoretical and simulated outcomes.
        \item Regression: Built predictive models and evaluated their performance.
        \item Hypothesis testing: Assessed differences using statistical tests.
        \item Linear regression assumptions: Ensured model reliability.
    \end{itemize}
    \item Emphasized critical thinking and interpretation of results.
\end{itemize}
\end{frame}

\frame{\titlepage}

\end{document}
